% handout
\documentclass[11pt,final,usepdftitle=false]{beamer}
\mode<presentation>
\usetheme{ln12en}
\graphicspath{{\logopath}{\picspath}{\figspath}{./logos/}{./pics/}{./figs/}}
\usepackage{listings}
%\usepackage{bera}
\usepackage{inconsolata}
\usepackage{moresize}
%\renewcommand*\ttdefault{cmvtt}


\hypersetup{pdftitle={Lucas Nussbaum - SSH}}
\title{An introduction to SSH}
\authorteaching
\instituteiutlp
\newcommand{\tilda}{\textasciitilde{}}
\date{}


\AtBeginSection[]
{
  \begin{frame}
    \frametitle{Plan}
    \tableofcontents[currentsection]
  \end{frame}
}

\usepackage{tikz}
\usetikzlibrary{shapes,arrows,positioning}
\newcommand{\Gentsroom}{{\fontfamily{mvs}\fontencoding{U}\fontseries{m}\fontshape{n}\selectfont\char120}}

\begin{document}

\frame{\titlepage}

\section{SSH basics}
\subsection{SSH 101}
\begin{frame}{Introduction}
\begin{itemize}
\item SSH = Secure SHell
\item Standard network protocol and service (TCP port 22)
\item Many implementations, including:
	\begin{itemize}
		\item OpenSSH: Linux/Unix, Mac OS X \alert{$\leftarrow$ this talk, mostly}
		\item Putty: Windows, client only
		\item Dropbear: small systems (routers, embedded)
	\end{itemize}
\item Unix command (\texttt{ssh}); server-side: \texttt{sshd}
\item Establish a \alert{secure communication channel} between two machines
\item Relies on cryptography
\item Most basic usage: \alert{get shell access} on a remote machine
\item Many advanced usages:
	\begin{itemize}
		\item Data transfer (\texttt{scp}, \texttt{sftp}, \texttt{rsync})
		\item Connect to specific services (such as Git or SVN servers)
		\item Dig secure tunnels through the public Internet
	\end{itemize}
\item Several authentication schemes: password, public key
\end{itemize}
\end{frame}

\begin{frame}{Basic usage}
\begin{itemize}
\item Connecting to a remote server:\\
	\texttt{\$ ssh login@remote-server} \\
	$\leadsto$ Provides a shell on \textsl{remote-server}

\hbr
\item Executing a command on a remote server:\\
	\texttt{\$ ssh login@remote-server ls /etc}

\hbr
\item Copying data (with \texttt{scp}, similar to \texttt{cp}):\\
	\texttt{\$ scp local-file login@remote-serv:remote-directory/} \\
	\texttt{\$ scp login@remote-serv:remote-dir/file local-dir/}\\
	Usual \texttt{cp} options work, e.g. \texttt{-r} (recursive)
\hbr
\item Copying data (with \texttt{rsync}, more efficient than \texttt{scp} with many files):\\
	\texttt{\$ rsync -avzP localdir login@server:path-to-rem-dir/}\\
	\hbr
	Note: trailing slash on source matters with \texttt{rsync} (not with \texttt{cp})\\
	\begin{itemize}
		\item \texttt{rsync -a dir1 u@h:dir2}  $\leadsto$ dir1 copied inside dir2
		\item \texttt{rsync -a dir1/ u@h:dir2}  $\leadsto$ content of dir1 copied to dir2
	\end{itemize}
\end{itemize}
\end{frame}

\subsection{Public-key authentication}
\begin{frame}{Public-key authentication}
\begin{itemize}
\item General idea:
	\begin{itemize}
		\item Asymmetric cryptography (or public-key cryptography):
			\begin{itemize}
				\item The public key is used to encrypt something
				\item Only the private key can decrypt it
			\end{itemize}
		\item User owns a private (secret) key, stored on the local machine
		\item The server has the public key corresponding to the private key 
		\item Authentication = \textsl{<server> prove that you own that private key!}
	\end{itemize}
\hbr
\item Implementation (\textsl{challenge-response authentication}):
	\begin{enumerate}
		\item Server generates a nonce (random value)
		\item Server encrypts the nonce with the Client's public key
		\item Server sends the encrypted nonce (= the challenge) to client
		\item Client uses the private key to decrypt the challenge
		\item Client sends the nonce (= the response) to the Server
		\item Server compares the nonce with the response
	\end{enumerate}
\end{itemize}
% FIXME schema
\end{frame}

\begin{frame}{Public-key authentication (2)}
\begin{itemize}
	\item Advantages:
		\begin{itemize}
			\item The password does not need to be sent over the network
			\hbr
			\item The private key never leaves the client
			\hbr
			\item The process can be automated
		\end{itemize}
	\hbr
	\item However, the private key should be protected (what if your laptop gets stolen?)
		\begin{itemize}
			\item Usually with a passphrase
		\end{itemize}
\end{itemize}
\end{frame}

\begin{frame}[fragile]{Key-pair generation}
\begin{lstlisting}[basicstyle=\ttfamily\scriptsize,escapeinside={||}]
$ ssh-keygen
Generating public/private rsa key pair.
Enter file in which to save the key (/home/user/.ssh/id_rsa): |\textbf{[ENTER]}|
Enter passphrase (empty for no passphrase): |\textbf{passphrase}|
Enter same passphrase again: |\textbf{passphrase}|
Your identification has been saved in /home/user/.ssh/id_rsa.
Your public key has been saved in /home/user/.ssh/id_rsa.pub.
The key fingerprint is:
f6:35:53:71:2f:ff:00:73:59:78:ca:2c:7c:ff:89:7b user@my.hostname.net
The key's randomart image is:
+--[ RSA 2048]----+
|              ..o|
|\textbf{(...)}|
|             .o  |
+-----------------+
$\end{lstlisting}
\begin{itemize}
	\item Creates the key-pair:
		\begin{itemize}
			\item \texttt{\tilda/.ssh/id\_rsa} (private key)
			\item \texttt{\tilda/.ssh/id\_rsa.pub} (public key)
		\end{itemize}
\end{itemize}
\end{frame}

\begin{frame}{Copying the public key to the server}
\begin{itemize}
\item Example public key:\\
	{\footnotesize \texttt{ssh-rsa AAAAB3NX[\ldots]hpoR3/PLlXgGcZS4oR user@my.hostname.net}}
\hbr

\item On the server, \texttt{\bf \tilda{}user/.ssh/authorized\_keys} contains
	the list of public keys authorized to connect to the \texttt{user} account

\hbr
\item The key can be copied manually there
\hbr
\item Or use \texttt{ssh-copy-id} to automatically copy the key:\\
	\texttt{client\$ ssh-copy-id user@server}
\hbr
\item Sometimes the public key needs to be provided using a web interface (e.g. on GitHub, FusionForge, Redmine, etc.)
\end{itemize}
\end{frame}

\begin{frame}{Avoiding typing the passphrase}
\begin{itemize}
\item If the private key is not protected with a passphrase, the connection is established immediately:\\ \hbr
{\ttfamily\footnotesize 
*** login@laptop:\tilda\$  ssh rlogin@rhost \textbf{ [ENTER]}\\
*** rlogin@rhost:\tilda\$\\ }
\hbr
\item Otherwise, \texttt{ssh} asks for the passphrase:\\ \hbr
{\ttfamily\footnotesize 
*** login@laptop:\tilda\$ ssh rlogin@rhost \textbf{ [ENTER]}\\
Enter passphrase for key '/home/login/id\_rsa':  \textbf{[passphrase+ENTER]}\\ 
*** rlogin@rhost:\tilda\$\\ }
\hbr
\item An \textbf{SSH agent} can be used to store the decrypted private key
\begin{itemize}
	\item Most desktop environments act as SSH agents automatically
	\item One can be started with \texttt{ssh-agent} if needed
	\item Add keys manually with \texttt{ssh-add}
\end{itemize}
\end{itemize}
\end{frame}

\subsection{Checking the server's identity}

\begin{frame}{Checking the server identity: \texttt{known\_hosts}}
\begin{itemize}
\item Goal: detect hijacked servers\\
\textsl{What if someone replaced the server to steal passwords?}
\hbr
\item When you connect to a server for the first time, \texttt{ssh} stores the server's public key in
	\texttt{\tilda/.ssh/known\_hosts}
\end{itemize}
{\ttfamily\small
*** login@laptop:\tilda\$ ssh rlogin@server\textbf{ [ENTER]}\\ 
The authenticity of host 'server (10.1.6.2)' can't be established.\\
RSA key fingerprint is 94:48:62:18:4b:37:d2:96:67:c9:7f:2f:af:2e:54:a5.\\
Are you sure you want to continue connecting (yes/no)? \textbf{yes [ENTER]}\\ 
Warning: Permanently added 'server,10.1.6.2'(RSA) to the list of known hosts.\\
rlogin@server's password:\\ }
\end{frame}

\begin{frame}[fragile]{Checking the server identity: \texttt{known\_hosts} (2)}
\begin{itemize}
\item During each following connection, \texttt{ssh} ensures that the key still matches, and warns the user otherwise
\end{itemize}
\begin{lstlisting}[basicstyle=\ttfamily\ssmall,escapeinside={||}]
*** login@laptop:~$ ssh rlogin@server|\textbf{ [ENTER]}|
@@@@@@@@@@@@@@@@@@@@@@@@@@@@@@@@@@@@@@@@@@@@@@@@@@@@@@@@@@@
@    WARNING: REMOTE HOST IDENTIFICATION HAS CHANGED!     @
@@@@@@@@@@@@@@@@@@@@@@@@@@@@@@@@@@@@@@@@@@@@@@@@@@@@@@@@@@@
IT IS POSSIBLE THAT SOMEONE IS DOING SOMETHING NASTY!
Someone could be eavesdropping on you right now (man-in-the-middle attack)!
It is also possible that a host key has just been changed.
The fingerprint for the RSA key sent by the remote host is
e3:94:03:90:5d:81:ed:bb:d5:d2:f2:de:ba:31:18:d8.
Please contact your system administrator.
Add correct host key in /home/login/.ssh/known_hosts to get rid of this message.
Offending RSA key in /home/login/.ssh/known_hosts:12
RSA host key for server has changed and you have requested strict checking.
Host key verification failed.
*** login@laptop:~$ 
\end{lstlisting}
\begin{itemize}
	\item A truly outdated key can be removed with\\ \texttt{ssh-keygen -R server}
\end{itemize}
\end{frame}

\subsection{Configuring SSH}
\begin{frame}{Configuring SSH}
	\begin{itemize}
		\item SSH gets configuration data from:
			\begin{enumerate}
				\item command-line options (\texttt{-o ...})
				\item the user's configuration file: \texttt{\tilda/.ssh/config}
				\item the system-wide configuration file: \texttt{/etc/ssh/ssh\_config}
			\end{enumerate}
		\hbr
		\item Options are documented in the \texttt{ssh\_config(5)} man page
		\hbr
		\item \texttt{\tilda/.ssh/config} contains a list of hosts (with wildcards)
		\hbr
		\item For each parameter, the first obtained value is used\\
			\begin{itemize}
				\item Host-specific declarations are given near the beginning
				\item General defaults at the end
			\end{itemize}
	\end{itemize}
\end{frame}

\begin{frame}[fragile]{Example: \texttt{\tilda/.ssh/config}}
\begin{lstlisting}[basicstyle=\ttfamily\small,escapeinside={||}]
Host mail.acme.com
    User root

Host foo # alias/shortcut. 'ssh foo' works
    Hostname very-long-hostname.acme.net
    Port 2222

Host *.acme.com
    User jdoe
    Compression yes # default is no
    PasswordAuthentication no # only use public key
    ServerAliveInternal 60 # keep-alives for bad firewall

Host *
    User john
\end{lstlisting}
\begin{itemize}
	\item Note: \texttt{bash-completion} can auto-complete using \texttt{ssh\_config} hosts
\end{itemize}
\end{frame}

\section{Advanced usage}

\subsection{SSH as a communication layer for applications}
\begin{frame}{SSH as a communication layer for applications}
	\begin{itemize}
		\item Several applications use SSH as their communication (and authentication) layer
		\hbr
		\item \texttt{scp}, \texttt{sftp}, \texttt{rsync} (data transfer)
			\begin{itemize}
				\item \texttt{lftp} (CLI) and \texttt{gftp} (GUI) support the SFTP protocol
			\end{itemize}
			\hbr
		\item \texttt{unison} (synchronization)
		\hbr

	\item \alert{Subversion:} \texttt{svn checkout svn+ssh://user@rhost/path/to/repo}
		\hbr
	\item \alert{Git:} \texttt{git clone ssh://git@github.com/path-to/repository.git}\\
	 Or: \texttt{git clone git@github.com:path-to/repository.git}
	\end{itemize}
\end{frame}

\subsection{Access remote filesystems over SSH: sshfs}
\begin{frame}[fragile]{Access remote filesystems over SSH: sshfs}
	\begin{itemize}
		\item \texttt{sshfs}: FUSE-based solution to access remote machines
			\hbr
	\item Ideal for remote file editing with a GUI, copying small amounts of data, etc.
			\hbr
		\item Mount a remote directory:\\
			\texttt{sshfs root@server:/etc /tmp/local-mountpoint}\\
			Unmount: \texttt{fusermount -u /tmp/local-mountpoint}
		\hbr
	\item Combine with \texttt{afuse} to auto-mount any machine:\\
	 \texttt{afuse -o mount\_template="sshfs \%r:/ \%m" -o \textbackslash \\
	 unmount\_template="fusermount -u -z \%m" \tilda/.sshfs/}\\[0.5em]
			$\leadsto$ \texttt{cd \tilda/.sshfs/rhost/etc/ssh}
	\end{itemize}
\end{frame}


\subsection{SSH tunnels, X11 forwarding, and SOCKS proxy}
\begin{frame}{SSH tunnels with -L and -R}
        \begin{itemize}
                \item Goal: transport traffic through a secure connection
			\begin{itemize}
				\item Work-around network filtering (firewalls)
				\item Avoid sending unencrypted data on the Internet
				\item But only works for TCP connections
			\end{itemize}
                        \hbr
		\item \textbf{-L}: access a remote service behind a firewall (Intranet server)
			\begin{itemize}
				\item \texttt{ssh -L 12345:service:1234 server}
				\item Still on \textsl{Client}: \texttt{telnet localhost 12345}
				\item \textsl{Server} establishes a TCP connection to \textsl{Service}, port 1234
				\item The traffic is tunnelled inside the SSH connection to \textsl{Server}
			\end{itemize}
	\end{itemize}
	
\begin{center}
\begin{tikzpicture}
	\node[name=internet,gray,cloud,aspect=1,cloud puffs=23,minimum width=5cm,minimum height=2cm,draw] at (0,0) {};
	\node[name=private,gray,cloud,aspect=1,cloud puffs=23,minimum width=5cm,minimum height=2cm,draw] at (5,0) {};
	\draw[blue,fill] (-2.5,-0.1) rectangle (2.5,0.1);
	\draw[darkred,fill] (7.5,0) circle (2mm);

	\draw[name=client,darkred,fill] (-2.5,0) circle (2mm);
	\draw[name=server,darkred,fill] (2.5,0) circle (2mm);
	\draw (-2.5,-1) node  {Client};
	\draw (2.5,-1) node  {Server};
	\draw (7.5,-1) node[align=center]  {TCP Service \\on port 1234};
	\draw (0, -0.2) node [below] {\small \sl Internet};
	\draw (5, -0.2) node [below] {\small \sl Private network};
	\only<2->{
	\draw[orange,fill,ultra thick,->] (-2.5,0) -- (7.3,0);
	\draw[name=client,darkred,fill] (-2.5,0) circle (2mm);
	\draw[name=server,darkred,fill] (2.5,0) circle (2mm);
}
\end{tikzpicture}
\end{center}
\end{frame}

\begin{frame}{SSH tunnels with -L and -R}
        \begin{itemize}
		\item \textbf{-R}: provide remote access to a local private service
			\begin{itemize}
				\item \texttt{ssh -R 12345:service:1234 server}
				\item On \textsl{Server}: \texttt{telnet localhost 12345}
				\item \textsl{Client} establishes a TCP connection to \textsl{Service}, port 1234
				\item The traffic is tunnelled inside the SSH connection to \textsl{Client}
			\end{itemize}
	\end{itemize}
	
\begin{center}
\begin{tikzpicture}
	\node[name=internet,gray,cloud,aspect=1,cloud puffs=23,minimum width=5cm,minimum height=2cm,draw] at (0,0) {};
	\node[name=private,gray,cloud,aspect=1,cloud puffs=23,minimum width=5cm,minimum height=2cm,draw] at (5,0) {};
	\draw[blue,fill] (2.5,-0.1) rectangle (7.5,0.1);

	\draw[name=client,darkred,fill] (-2.5,0) circle (2mm);
	\draw[name=server,darkred,fill] (2.5,0) circle (2mm);
	\draw[darkred,fill] (7.5,0) circle (2mm);
	\draw (2.5,-1) node  {Client};
	\draw (7.5,-1) node  {Server};
	\draw (-2.5,-1) node[align=center]  {TCP Service \\on port 1234};
	\draw (0, -0.2) node [below] {\small \sl Private network};
	\draw (5, -0.2) node [below] {\small \sl Internet};
	\only<2->{
	\draw[orange,fill,ultra thick,->] (7.5,0) -- (-2.3,0);
	\draw[darkred,fill] (2.5,0) circle (2mm);
	\draw[darkred,fill] (7.5,0) circle (2mm);
}
\end{tikzpicture}
\end{center}
\begin{itemize}
	\item Notes:
		\begin{itemize}
			\item SSH tunnels don't work very well for HTTP, because IP+port are not enough to identify a website (\texttt{Host:} HTTP header)
			\item By default, tunnels are only bound to localhost. Use \texttt{-g} (gateway) to allow remote hosts to connect to local forwarded ports
		\end{itemize}
\end{itemize}
\end{frame}

\begin{frame}{X11 forwarding with -X: GUI apps over SSH}
\begin{itemize}
	\item Run a graphical application on a remote machine, display locally
		\hbr
	\item Similar to VNC, but on a per-application basis
		\hbr
	\item \texttt{ssh -X server}
		\hbr
	\item \texttt{\$DISPLAY} will be set by SSH on the server:\\
		\texttt{\$ echo \$DISPLAY\\
		localhost:10.0}
		\hbr
\item Then start GUI applications on server (e.g. \texttt{xeyes})
		\hbr
	\item Troubleshooting:
		\begin{itemize}
			\item \texttt{xauth} must be installed on the remote machine
			\item The local Xorg server must allow TCP connections
				\begin{itemize}
					\item \texttt{pgrep -a Xorg} $\leadsto$ \texttt{-nolisten} must not be included 
					\item Can be configured in your login manager
				\end{itemize}
			\item Does not work very well over slow or high-latency connections
		\end{itemize}
\end{itemize}
\end{frame}

\begin{frame}[fragile]{SOCKS proxy with -D}
\begin{itemize}
\item SOCKS: protocol to proxy TCP connections via a remote machine
\hbr
\item SSH can act as a SOCKS server: \texttt{ssh -D 1080 server}
\hbr
\item Use case similar to tunnelling with -L, but more flexible
	\begin{itemize}
		\item Set up the proxy once, use for multiple connections
	\end{itemize}
\hbr
\item Usage:
\begin{itemize}
\item Manual: configure applications to use the SOCKS proxy
\hbr
\item Transparent: use \texttt{tsocks} to re-route connections via SOCKS
\end{itemize}
\begin{lstlisting}[basicstyle=\ttfamily\small,escapeinside={||}]
$ cat /etc/tsocks.conf
server = 127.0.0.1
server_type = 5
server_port = 1080 # then start ssh with -D 1080
$ tsocks pidgin # tunnel application through socks
\end{lstlisting}
\begin{itemize}
	\item Another transparent proxifier is \texttt{redsocks} (uses \texttt{iptables} rules to redirect to a local daemon instead of \texttt{LD\_PRELOAD})
\end{itemize}
\end{itemize}
\end{frame}

\subsection{VPN over SSH}
\begin{frame}{VPN over SSH}
	\begin{itemize}
		\item Built-in support for tun-based VPN
			\hhbr
			\begin{itemize}
				\item \texttt{PermitTunnel yes} required on server (disabled by default)
					\hbr
				\item \texttt{ssh -w 0:0 root@server} (0, 0 are the tun device numbers)
					\hbr
				\item Then configure IP addresses and routing on both sides
			\end{itemize}
			\br
		\item \texttt{sshuttle}: another VPN over SSH solution
			\hhbr
			\begin{itemize}
				\item Root access not required on the server side
					\hbr
				\item Idea similar to \texttt{slirp}
					\hbr
				\item Uses \texttt{iptables} rules to redirect traffic to VPN
					\hbr
				\item (as root:) \texttt{sshuttle -r user@server 0/0 -vv}
					\hbr
				\item Limitation: does not support tunnelling UDP or ICMP traffic
			\end{itemize}
	\end{itemize}
\end{frame}

\subsection{Jumping through hosts with ProxyCommand}
\begin{frame}[fragile]{Jumping through hosts with ProxyCommand}

\begin{tikzpicture}
	\node[name=internet,gray,cloud,aspect=1,cloud puffs=23,minimum width=5cm,minimum height=2cm,draw] at (0,0) {};
	\node[name=private,gray,cloud,aspect=1,cloud puffs=23,minimum width=5cm,minimum height=2cm,draw] at (5,0) {};
	\draw[name=client,darkred,fill] (-2.5,0) circle (2mm);
	\draw[name=server,darkred,fill] (2.5,0) circle (2mm);
	\draw[darkred,fill] (7.5,0) circle (2mm);
	\draw (-2.5,-1) node  {Client};
	\draw (2.5,-1) node[align=center]  {Server\\(gateway, firewall, etc.)};
	\draw (7.5,-1) node[align=center]  {Server2 on\\ private network};
	\draw (5, -0.2) node [below] {\small \sl Private network};
	\draw (0, -0.2) node [below] {\small \sl Internet};
	\draw[orange,fill,ultra thick,->] (-2.3,0) -- (2.3,0);
	\draw[orange,fill,ultra thick,->] (2.7,0) -- (7.3,0);
\end{tikzpicture}
\begin{itemize}
\item Problem: to connect to Server2, you need to connect to Server
	\begin{itemize}
		\item Can you do that in a single step? (required for data transfer, tunnels, X11 forwarding)
	\end{itemize}
	\hbr
\item Combines two SSH features:
\begin{itemize}
\item \texttt{ProxyCommand} option: command used to connect to host; connection available on standard input \& output
\hbr
\item \texttt{ssh -W host:port} $\leadsto$ establish a TCP connection, provide it on standard input \& output (suitable for \texttt{ProxyCommand})
\end{itemize}
\end{itemize}
\end{frame}
\begin{frame}[fragile]{Jumping through hosts with ProxyCommand (2)}
	\hbr
\begin{itemize}
\item Example configuration:
\end{itemize}\vspace{-0.8em}
\begin{lstlisting}[basicstyle=\ttfamily\small,escapeinside={||}]
Host server2 # ssh server2 works
    ProxyCommand ssh -W server2:22 server
\end{lstlisting}\vspace{-0.2em}
\hbr
\begin{itemize}
\item Also works with wildcards
\end{itemize}\vspace{-0.8em}
\begin{lstlisting}[basicstyle=\ttfamily\small,escapeinside={||}]
Host *.priv # ssh host1.priv works
    ProxyCommand ssh -W $(basename %h .priv):%p server
\end{lstlisting}\vspace{-0.2em}
\begin{itemize}
	\item \texttt{-W} only available since OpenSSH 5.4 (circa 2010), but the same can be achieved with \texttt{netcat}:
\end{itemize}\vspace{-0.8em}
\begin{lstlisting}[basicstyle=\ttfamily\small,escapeinside={||}]
Host *.priv
    ProxyCommand ssh serv nc -q 0 $(basename %h .priv) %p
\end{lstlisting}\vspace{-0.2em}
\begin{itemize}
\item Similar solution to connect via a proxy:
\begin{itemize}
	\item SOCKS: \texttt{connect-proxy -4 -S myproxy:1080 rhost 22}
	\item HTTP (with CONNECT): \texttt{corkscrew myproxy 1080 rhost 22}
	\item When CONNECT requests are forbidden, set up \texttt{httptunnel} on a remote server, and use \texttt{htc} and \texttt{hts}
\end{itemize}
\end{itemize}
\end{frame}

\subsection{Triggering remote command execution securely}
\begin{frame}[fragile]{Triggering remote command execution securely}
\begin{itemize}
\item Goal: notify Server2 that something finished on Server1
	\begin{itemize}
		\item But Server1 must not have full shell access on Server2
	\end{itemize}
	\hbr
\item Method: limit to a single command in \texttt{authorized\_keys}
	\begin{itemize}
		\item Also known as SSH triggers
	\end{itemize}
\hbr
\item Example \texttt{authorized\_keys} on Server2:
\end{itemize}
\begin{lstlisting}[basicstyle=\ttfamily\small,escapeinside={||}]
from="server1.acme.com",command="tar czf - /home",no-pty,
no-port-forwarding ssh-rsa AAAA[...]oR user@my.host.net
\end{lstlisting}
\end{frame}

\subsection{Escape sequences}
\begin{frame}[fragile]{Escape sequences}
\begin{itemize}
\item Goal: interact with an already established SSH connection
	\begin{itemize}
		\item Add tunnels or SOCKS proxy, kill unresponsive connection
	\end{itemize}
\hbr
\item Escape sequences start with '\tilda', at the beginning of a line
	\begin{itemize}
		\item So press \texttt{[enter]}, then \texttt{\tilda}, then e.g. '\texttt{?}'
	\end{itemize}
\hbr
\item Main sequences (others documented in \texttt{ssh(1)}):
	\begin{itemize}
		\item \texttt{\tilda{}.} -- disconnect (for unresponsive connections)
		\item \texttt{\tilda{}?} -- show the list of escape sequences
		\item \texttt{\tilda{}C} -- open SSH command-line. e.g. \texttt{\tilda{}C -D 1080}
		\item \texttt{\tilda{}\&} -- logout and background SSH while waiting for forwarded connections or X11 sessions to terminate
	\end{itemize}
\end{itemize}
\end{frame}

\subsection{Related tools}
\begin{frame}{Related tools}
\begin{itemize}
\item \texttt{screen} and \texttt{tmux}: provide virtual terminals on remote
	machines where you can start long-running commands, disconnect, and
	reconnect later
	\hbr
\item \texttt{mosh}: SSH alternative suited for Wi-Fi, cellular and long distance links
	\hbr
\item \texttt{autossh}: checks an SSH session every 10 minutes, and restart it if needed\\
	\texttt{autossh -t server 'screen -RD'}: maintain a screen session open despite network disconnections
\end{itemize}
\end{frame}

\section{Conclusions}
\begin{frame}{Conclusions}
	\begin{itemize}
		\item The Swiss-army knife of remote administration
			\hbr
		\item Very powerful tool, many useful features
			\hbr
		\item Practical session: test everything mentioned in this presentation
			\begin{enumerate}
				\item \texttt{scp}, \texttt{rsync}
				\item Key-based authentication
				\item Using an SSH agent
				\item aliases in SSH configuration
				\item \texttt{sshfs}, \texttt{sftp}
				\item SSH tunnels
				\item X11 forwarding
				\item SOCKS proxy with \texttt{tsocks}
				\item Jumping through hosts
				\item Escape sequences
				\item \ldots
			\end{enumerate}
	\end{itemize}
\end{frame}

\end{document}
